\documentclass [a4paper,11pt]{book}
\usepackage [T1] {fontenc}		% codifica dei font
\usepackage [utf8] {inputenc}		% lettere accentate da tastiera
\usepackage [italian] {babel}		% lingua del documento
\usepackage {url}					% per scrivere gli indirizzi Internet
\usepackage{graphicx}
\usepackage{latexsym}
\usepackage{listings}
\usepackage{xcolor}

%Stile display del codice
\definecolor{codegreen}{rgb}{0,0.6,0}
\definecolor{codegray}{rgb}{0.5,0.5,0.5}
\definecolor{codepurple}{rgb}{0.58,0,0.82}
\definecolor{backcolour}{rgb}{0.95,0.95,0.92}

\lstdefinestyle{mystyle}{
    backgroundcolor=\color{backcolour},   
    commentstyle=\color{codegreen},
    keywordstyle=\color{magenta},
    numberstyle=\tiny\color{codegray},
    stringstyle=\color{codepurple},
    basicstyle=\ttfamily\footnotesize,
    breakatwhitespace=false,         
    breaklines=true,                 
    captionpos=b,                    
    keepspaces=true,                 
    numbers=left,                    
    numbersep=5pt,                  
    showspaces=false,                
    showstringspaces=false,
    showtabs=false,                  
    tabsize=2
}

\lstset{style=mystyle}
%fine stile
\begin{document}

\author{Luca Zanchetta, Mattia Aquilina}
\title{Tesina LWEB - neWinfoStud}
\maketitle
\
\tableofcontents

\chapter{Introduzione e descrizione dei dati}

\section{Specifiche introduttive}

Si propone di realizzare una piattaforma che replichi le funzionalità del noto portale infostud. Il sito è visitabile a partire da una pagina home, accessibile da tutte le utenze, che illustra i corsi di laurea offerti dall’università con annesse informazioni relative agli esami che essi prevedono.
  
La lista dei corsi di laurea e degli esami può essere visionata applicando un filtro di visualizzazione basato sul nome.
Tramite un’apposita pagina di login, il visitatore può autenticarsi utilizzando le proprie credenziali, accedendo così alle funzionalità competenti alla sua utenza.

Un utente che sia autenticato come studente può innanzitutto visualizzare tutte le proprie informazioni personali (elencate successivamente). 

Tramite due pagine dedicate, egli può inoltre prenotarsi o cancellare le prenotazioni degli appelli. Oltre a ciò, uno studente può accedere e contribuire alla bacheca dei corsi. In particolare, esiste una bacheca per ciascun corso, all’interno della quale gli studenti possono creare dei post o contribuire a post già creati. 

Gli stessi studenti, e non solo i creatori dei post, possono valutare le risposte ottenute secondo i metodi di giudizio di utilità e accordo. L’insieme dei giudizi ricevuti costituirà la reputazione di uno studente.

Un’altra funzionalità importante è quella del FAQ. Esiste una pagina di FAQ per ogni corso. Queste pagine possono essere costruite dai professori aggiungendo sia coppie di domande e risposte, sia risposte a domande proposte dagli studenti.

In entrambe le due pagine (Bacheca e FAQ), un’utenza autenticata come Segreteria è in grado di moderare, potendo modificare, eliminare o fissare post e domande.

Le tipologie di utenti che possono usufruire della piattaforma, con annesse rispettive funzionalità, sono le seguenti:
\begin{itemize}
\item Visitatore (utente non autenticato);
\item Studente;
\item Professore;
\item Segreteria (moderatore FAQ);
\item Utente Amministratore.
\end{itemize}

\medskip
\medskip
\medskip

Le funzionalità associate alle utenze sopra elencate sono elencate di seguito.
\begin{description}
\item[Utente visitatore.] Un utente di questo tipo è abilitato a:

\begin{itemize}
\item Visualizzare le informazioni relative a corsi di laurea ed esami associati;
\item Accedere alla form di login;
\item Registrarsi presso la piattaforma.
\end{itemize}

\item[Utente registrato – Studente.] Un utente di questo tipo è abilitato a:

\begin{itemize}
\item Visualizzare informazioni riguardanti la SUA carriera (reputazione acquisita, media, esami superati, cfu acquisiti, ecc…);
\item Prenotarsi agli appelli dei corsi relativi al suo percorso di studi;
\item Contribuire alla bacheca dei corsi, uno spazio dove più utenti possono discutere di argomenti riguardanti il singolo corso;
\item Recensire i post/commenti degli altri studenti;
\item Visualizzare le pagine FAQ dei corsi, ed eventualmente contribuire con delle domande.
\end{itemize}

\item[Utente registrato – Professore.] Un utente di questo tipo è abilitato a:

\begin{itemize}
\item Aggiungere, modificare o eliminare appelli del suo corso;
\item Visualizzare gli studenti prenotati ad un dato appello, ed eventualmente registrare i voti, rinunce o bocciature;
\item Utilizzare il portale delle FAQ inserendo nuove domande con risposte, oppure rispondendo alle domande che gli studenti propongono.
\end{itemize}

\item[Utente registrato – Segreteria.] Un utente di questo tipo è abilitato a:

\begin{itemize}
\item Moderare la bacheca dei corsi;
\item Moderare le pagine FAQ, in particolare fissando alcune risposte o inserendone di nuove, che in caso saranno fissate in cima;
\item Visualizzare la lista degli appelli proposti da tutti i professori;
\item Inserire, modificare ed eliminare appelli relativi agli esami presenti.
\end{itemize}

\item[Utente Amministratore.] Un utente di questo tipo è abilitato a:

\begin{itemize}
\item Sospendere o bloccare utenti;
\item Moderare le sezioni bacheca e FAQ, potendone eliminare i post, o alcune risposte;
\item Modificare dati associati delle altre utenze;
\item Inserire, modificare ed eliminare appelli relativi agli esami presenti.
\end{itemize}
\end{description}

\medskip
\medskip

I dati che verranno manipolati dalla piattaforma sono elencati di seguito.
\begin{description}
\item[Studente.] Verranno manipolati i seguenti dati:

\begin{itemize}
\item Matricola;
\item Nome;
\item Cognome;
\item Data di nascita;
\item Crediti acquisiti;
\item Media;
\item Lista esami prenotati; 
\item Corso di laurea;
\item Lista esami sostenuti;
\item Reputazione Totale;
\item Password di accesso.
\end{itemize}

\item[Professore.] Verranno manipolati i seguenti dati:

\begin{itemize}
\item Nome;
\item Cognome;
\item Matricola;
\item Corso;
\item Password di accesso.
\end{itemize}

\item[Segreteria.] Verranno manipolati i seguenti dati:
\begin{itemize}
\item Username e password di accesso al pannello di amministrazione.
\end{itemize}
\end{description}

\medskip
\medskip

\section{Funzionalità specifiche}

In questa sezione verranno approfondite le funzionalità elencate nella sezione precedente.

\subsection{Funzionalità visitatore}

\subsubsection{Login e registrazione}

La piattaforma, tramite un'opportuna form, offre la possibilità all'utente di effettuare il login, in modo da poter accedere alle funzionalità previste per la sua tipologia di utenza. In particolare, poiché ciascuna utenza possiede caratteristiche differenti, al fine di permetterne il login viene inserito un menù di scelta multipla, che consentirà all'utente di scegliere l'utenza con cui desidera autenticarsi. In base alla scelta effettuata dall'utente, comparirà una diversa form di login; per uno studente e per un professore verranno richiesti matricola e password; per un addetto alla segreteria e per un utente amministratore, invece, verranno richiesti uno username e una password.

Qualora l'utente visitatore non fosse registrato presso il portale, invece, potrà sempre avere la possibilità di registrarsi, specificando la tipologia di utenza desiderata. In base alla tipologia di utenza scelta, dunque, verrà presentata un'opportuna form di registrazione, che richiederà l'inserimento dei dati già presentati nel \emph{paragrafo 1.1}.

\medskip

\subsubsection{Visualizzazione di corsi ed esami}

La piattaforma mette a disposizione dell'utente, sia egli autenticato oppure un semplice visitatore, la possibilità di visualizzare un elenco dei corsi di laurea offerti dall'università cui il sito fa riferimento. Tale elenco può essere presentato all'utente in vari modi. In particolare, è data la possibilità all'utente di visualizzare l'elenco dei corsi di laurea per intero (opzione di default, raggiungibile tramite la pressione di un apposito pulsante), o ridotto, tramite un filtro di ricerca basato sul nome. Questo filtro verrà implementato tramite una barra di ricerca, nella quale l'utente può inserire del contenuto testuale; la ricerca potrà concludersi con esito negativo o con esito positivo. In caso di esito negativo, verrà semplicemente stampato un opportuno messaggio di errore, indicante il fatto che non è presente nella piattaforma un corso di laurea il cui nome soddisfi le specifiche richieste dall'utente. In caso di esito positivo, invece, verrà presentato un elenco contenente tutte le voci (almeno una) relative ai corsi di laurea il cui nome contiene, almeno in parte, il contenuto testuale inserito dall'utente.

Una volta presentato il suddetto elenco, l'utente può selezionare una delle voci che compaiono a schermo; in questo caso, si avrà accesso ad una pagina di visualizzazione degli esami relativi al particolare corso di laurea selezionato. La presentazione degli esami avviene secondo modalità analoghe a quelle descritte per la presentazione dei corsi di laurea; inizialmente, vengono presentati a schermo tutti gli esami in appositi riquadri, ciascuno contenente il nome dell'esame e un pulsante \emph{info} che, se premuto, reindirizza l'utente ad una pagina di informazioni relative all'esame selezionato. Anche nel caso degli esami, inoltre, è possibile sfruttare la barra di ricerca per imporre un filtro sul nome dell'esame, con modalità identiche a quelle precedentemente esplicate.

\medskip

\subsection{Funzionalità studente}

\subsubsection{Visualizzazione informazioni personali}

All'interno della piattaforma l'utenza studente è definita da molti campi, alcuni dei quali rappresentano la carriera, mentre altri semplici dati anagrafici. Il portale mette a disposizione diverse pagine per permettere la visualizzazione e la modifica, ove fosse permessa, di questi ultimi.

Tramite un'apposita pagina è possibile visualizzare tutte le informazioni relative alla propria carriera. In particolare, potrà visualizzare: il totale dei crediti acquisiti, media ponderata e aritmetica, reputazione totale (si veda sezione bacheca). Strettamente legato alla carriera, esiste uno storico di esami, anch'esso visualizzabile tramite apposita pagina, che consente all'utente di ripercorre tutti gli esami dati, con informazioni relative a voto e data. La navigazione di questo elenco viene facilitata dalla possibilità di effettuare una ricerca tramite nome, o imponendo un diverso un ordinamento dei contenuti, scegliendo tra quelli proposti: per voto (crescente o decrescente), per data (crescente o decrescente), ordine alfabetico.

Una terza pagina, invece, permette all'utente di visualizzare le proprie informazioni personali quali nome, cognome, matricola e data di nascita. Si permette la modifica del solo campo della password.

\medskip

\subsubsection{Gestione delle prenotazioni degli appelli}

Il sistema degli appelli è una delle funzionalità base previste dalla piattaforma. Per \emph{appello} si intende lo svolgimento (solo figurativo) dell'esame di un determinato corso. Ogni appello viene inserito nella piattaforma, generalmente da un professore; tale funzionalità in ogni caso è accessibile anche alla segreteria e all'utente amministratore per motivi logistici di gestione. In questa sezione tratteremo solamente il punto di vista degli studenti.

Uno studente, potendo visualizzare tutti gli appelli relativi al corso a cui è iscritto, ha la possibilità di prenotare la propria partecipazione allo svolgimento del relativo l'esame. Qualora sorgesse la necessità, è sempre possibile cancellare una prenotazione effettuata.

\medskip

\subsubsection{Utilizzo della bacheca dei corsi}

Una bacheca è uno spazio che permette a diversi studenti di scambiarsi informazioni. Tale scambio di informazioni, che può essere volto all'aiuto reciproco, può avvenire tramite la pubblicazione di post o tramite replica a post già esistenti.

Una bacheca contiene tutti i post, con i relativi commenti, riguardanti un singolo esame di un corso di laurea; di conseguenza, la piattaforma mette a disposizione una bacheca per ogni esame di ogni corso di laurea. Con il termine \emph{post} si intende un messaggio testuale, con funzione di opinione, commento o intervento, inviato nella relativa bacheca. Tutti i post di una bacheca saranno presentati all'utente sotto forma di riquadro testuale, il cui contenuto rappresenta l'effettiva informazione; inoltre, un post è caratterizzato anche dal suo autore e dall'istante temporale di pubblicazione. Un \emph{commento} è un messaggio testuale di intervento in relazione al post cui si riferisce; viene visualizzato in subordinazione al post cui si riferisce, ed è caratterizzato da un contenuto testuale, che rappresenta l'informazione di intervento, dal suo autore e dall'istante temporale di pubblicazione. Ogni post ed ogni commento, infine, possono essere recensiti e giudicati da altri studenti in base a criteri di utilità e accordo, che saranno definiti nel seguito del paragrafo.

Uno studente, qualora avesse la necessità di ricavare informazioni aggiuntive su un singolo esame, può scegliere di visitare l'apposita bacheca, nella quale saranno contenuti i post e relativi commenti riguardanti l'esame in oggetto. Nel caso in cui le richieste dello studente non risultassero soddisfatte dai post esistenti nella bacheca, lo studente in questione può decidere di creare egli stesso un post, in cui avanzare esplicitamente la richiesta. Tale post potrà poi essere commentato da altri studenti. Nel caso in cui, invece, lo studente si sentisse preparato per rispondere ad una domanda presente in un post già esistente in bacheca, egli può scegliere di commentare quello stesso post, inserendo la risposta che ritiene più opportuna.

\medskip

Per quanto riguarda la fruizione dei contenuti delle bacheche, il portale mette a disposizione una pagina web che permette la visualizzazione di tutte le bacheche esistenti. Tale pagina offre inoltre una funzionalità di ricerca di una singola bacheca tramite un filtro applicato sul nome; l'implementazione logica della ricerca è la stessa già precedentemente esplicata nel caso della ricerca di corsi dei laurea o degli esami. Selezionando una particolare bacheca, è possibile accedere ad un'altra pagina web che permette la visualizzazione di un elenco di tutti i titoli dei post contenuti nella bacheca selezionata. Selezionando un particolare post fra quelli elencati, l'utente viene reindirizzato verso una pagina web che permette la visualizzazione completa del post, in cima alla pagina stessa, a cui seguono gli eventuali commenti pubblicati da altri studenti. 

\medskip

\subsubsection{Recensione di post e commenti}

Ogni post ed ogni commento contenuti nelle varie bacheche, come già precedentemente accennato, possono essere recensiti e/o giudicati da altri studenti in base a criteri di utilità e accordo. Per \emph{accordo} si intende il grado di adeguatezza del commento rispetto al contesto individuato dal post. Per \emph{utilità}, invece, si intende il grado di interesse della community relativo ad un post, ed in particolare al contesto di cui esso fa parte.

La recensione di un commento viene effettuata tramite un meccanismo di valutazione basato sull'assegnazione di un punteggio, che varia in valore da 1 a 5 punti. Il singolo commento, pertanto, sarà caratterizzato da un grado di accordo pari alla media delle valutazioni ricevute.

La recensione di un post, invece, viene effettuata tramite un meccanismo di valutazione basato sull'assegnazione di un \emph{up} o di un \emph{down}, che incrementano o decrementano rispettivamente un punteggio complessivo; questo punteggio complessivo andrà a caratterizzare l'utilità del post stesso.

\medskip

\subsubsection{Utilizzo del sistema delle FAQ}

Il sistema delle FAQ è un potente strumento che permette ai professori di andare incontro alle necessità dei loro studenti. Infatti, questa funzionalità prevede un sistema di domande e risposte, che tentanto di ricoprire tutti le possibili perplessità in relazione al corso trattato. Tratteremo ora solamente la fruizione di questo strumento dal punto di vista dello studente.

Uno studente può accedere ad una homepage dedicata contente l'elenco di tutte le FAQ associate agli esami del suo corso di laurea. Ciascun elemento reindirizza lo studente ad una pagina che permette di visualizzare l'elenco delle FAQ relative all'esame selezionato. Quest'ultima pagina è costituita da due blocchi, disposti in colonna: il primo situato nella parte alta della pagina contente le FAQ complete di domanda e risposta; il secondo invece, conterrà quelle FAQ, proposte dagli studenti, a cui non è ancora stata data una risposta.

Lo studente, oltre a poter visualizzare il semplice elenco delle FAQ, ha la possibilità di proporre nuove domande al docente, o votare, tramite un sistema basato sull'\emph{utilità}, analogo a quello già precedentemente esplicato. 

\emph{Si fa notare al lettore che lo studente \textbf{NON} è in grado di rispondere ad una domanda proposta in una FAQ; tale compito è di competenza del docente.}

\medskip

\subsection{Funzionalità professore}

\subsubsection{Gestione degli appelli ed esami}

Il sistema degli appelli è una delle funzionalità base previste dalla piattaforma. Per \emph{appello} si intende lo svolgimento (solo figurativo) dell'esame di un determinato corso. Ogni appello viene inserito nella piattaforma, generalmente da un professore; tale funzionalità in ogni caso è accessibile anche alla segreteria e all'utente amministratore per motivi logistici di gestione. In questa sezione tratteremo solamente il punto di vista dei docenti.

Un docente è in grado di inserire, modificare o eliminare un appello relativo al corso a lui referenziato. L'inserimento richiede che vengano specificate la data e l'ora dell'appello. 
Un docente, può inoltre gestire l'esame vero e proprio; in particolare è in grado di visualizzare gli studenti prenotati a ciascun appello, ed eventualmente inserire la votazione conseguita da ciascuno studente. La votazione può, o essere espressa da un giudizio compreso tra 18/30 e 30/30, oppure espressa tramite lettera, R o B; la lettera R indica il ritiro dello studente dall'esame, mentre la lettera B indica una bocciatura.

\medskip

\subsubsection{Utilizzo del sistema delle FAQ}

Il sistema delle FAQ è un potente strumento che permette ai professori di andare incontro alle necessità dei loro studenti. Infatti, questa funzionalità prevede un sistema di domande e risposte, che tentanto di ricoprire tutti le possibili perplessità in relazione al corso trattato. Tratteremo ora solamente la fruizione di questo strumento dal punto di vista del docente.

A differenze di uno studente, ad un docente è data la possibilità di inserire coppie domanda-risposta all'interno della FAQ relativa al proprio corso o, in alternativa, la possibilità di completare le domande proposte dagli studenti.

\emph{Si fa notare al lettore che un docente è in grado di visualizzare la sola FAQ di sua competenza, ossia la FAQ dell'unico corso assegnatogli.}
\medskip

\subsection{Funzionalità segreteria}

\subsubsection{Moderazione delle bacheche}

Una bacheca è uno spazio che permette a diversi studenti di scambiarsi informazioni. Tale scambio di informazioni, che può essere volto all'aiuto reciproco, può avvenire tramite la pubblicazione di post o tramite replica a post già esistenti. Tratteremo ora solamente la fruizione di questo strumento dal punto di vista della segreteria.

Per \emph{moderazione} delle bacheche si intende la possibilità di controllare i contenuti testuali di ciascun post e commento di ciascuna bacheca. Qualora fosse necessario intervenire, è sempre possibile eliminare uno o più commenti o un intero post. 

\medskip

\subsubsection{Moderazione delle FAQ}

Il sistema delle FAQ è un potente strumento che permette ai professori di andare incontro alle necessità dei loro studenti. Infatti, questa funzionalità prevede un sistema di domande e risposte, che tentanto di ricoprire tutti le possibili perplessità in relazione al corso trattato. Tratteremo ora solamente la fruizione di questo strumento dal punto di vista della segreteria.

La segreteria è incaricata di moderare la sezione delle FAQ. In particolare, per \emph{moderazione}, in questo caso, si intende la possibilità di controllare i contenuti di tutte le FAQ. In caso di necessità, essa può intervenire direttamente sulle singole FAQ, potendone modificare o eliminare il contenuto. 


\medskip

\subsubsection{Gestione degli appelli}

Il sistema degli appelli è una delle funzionalità base previste dalla piattaforma. Per \emph{appello} si intende lo svolgimento (solo figurativo) dell'esame di un determinato corso. Ogni appello viene inserito nella piattaforma, generalmente da un professore; tale funzionalità in ogni caso è accessibile anche alla segreteria e all'utente amministratore per motivi logistici di gestione. In questa sezione tratteremo solamente il punto di vista della segreteria.

La segreteria, per ragioni amministrative può accedere alle funzionalità, sia degli studenti, sia dei docenti.
\medskip

\subsection{Funzionalità amministratore}

\subsubsection{Gestione degli account utenti}

La gestione degli account è una funzionalità strettamente riservata all'utenza  amministratore. Consiste nel poter visualizzare, ed eventualmente modificare, le informazioni relative ad ogni tipo di account e, qualora ritenesse necessario un intervento diretto su un particolare account, può decidere di optare per una sospensione o eliminazione dello stesso.

\emph{Si fa notare al lettore che è sempre possibile revocare una sospensione.} 

Più nel dettaglio, esiste una schermata per ogni tipo di utenza, che permette all'amministratore di visualizzare tutte le utenze registrate presso il portale, appartenenti al tipo che si sta visualizzando. La navigazione all'interno di queste pagine è facilitata dalla possibilità di effettuare ricerche in base al campo identificativo dell'utenza. 
A partire da una di queste schermate, e premendo sull'utenza interessata, si accede ad una pagina che permette la visualizzazione, e l'eventuale modifica, delle informazioni relative ad essa. Inoltre, attraverso la pagina in questione, è possibile sospendere (\emph{Soft Ban}) o eliminare (\emph{Ban definitivo}).

\medskip

\emph{Si fa notare al lettore che l'attuazione di una sospensione (\emph{Soft Ban}) su un utente, non gli impedisce l'accesso a tutta la piattaforma, ma solamente alla sezione bacheca e alle sezione FAQ.}


\medskip

\subsubsection{Moderazione delle bacheche}

Una bacheca è uno spazio che permette a diversi studenti di scambiarsi informazioni. Tale scambio di informazioni, che può essere volto all'aiuto reciproco, può avvenire tramite la pubblicazione di post o tramite replica a post già esistenti. Tratteremo ora solamente la fruizione di questo strumento dal punto di vista della amministrazione.

L'utente amministratore è in grado di usufruire di tutte le funzionalità disponibili per tutte le utenze relative alla bacheca sopra esplicate.

\medskip

\subsubsection{Moderazione delle FAQ}

Il sistema delle FAQ è un potente strumento che permette ai professori di andare incontro alle necessità dei loro studenti. Infatti, questa funzionalità prevede un sistema di domande e risposte, che tentanto di ricoprire tutti le possibili perplessità in relazione al corso trattato. Tratteremo ora solamente la fruizione di questo strumento dal punto di vista dell'amministratore.

L'utente amministratore, oltre alle funzionalità accessibili dall'utenza  \emph{segreteria} (già trattate nel \emph{paragrafo 1.2.4}), può forzare l'inserimento di coppie domande-risposta.

\medskip

\subsubsection{Gestione degli appelli}

Il sistema degli appelli è una delle funzionalità base previste dalla piattaforma. Per \emph{appello} si intende lo svolgimento (solo figurativo) dell'esame di un determinato corso. Ogni appello viene inserito nella piattaforma, generalmente da un professore; tale funzionalità in ogni caso è accessibile anche alla segreteria e all'utente amministratore per motivi logistici di gestione. In questa sezione tratteremo solamente il punto di vista dell'utente amministratore.

L'utente amministratore, per ragioni amministrative, può accedere alle funzionalità sia degli studenti, sia dei docenti.

\medskip

\section{Strutture dati}
In questa sezione verranno approfonditi alcuni aspetti della piattaforma. Partiremo con l'indicare alcuni esempi di utenza che potranno essere presenti nel portale; questi esempi sono riportati nelle tabelle sottostanti.

\medskip

\subsection{Anagrafica}

\textbf{Studente}

\medskip

\resizebox{\textwidth}{!}{\begin{tabular}{|c|c|c|c|c|c|c|c|c|}
\hline
Matricola & Nome & Cognome & Password & DataNascita & ReputazioneTotale & CFU & Media & CorsoLaurea\\
\hline
1 & Mario & Rossi & supermario & 01/01/2000 & 0 & 0 & 0 & 1\\
\hline
\end{tabular}}

\medskip

\textbf{Docente}

\medskip

\begin{tabular}{|c|c|c|c|c|}
\hline
Matricola & Nome & Cognome & Password & idCorso\\
\hline
1 & Umberto & Nanni & meglio che niente & 1\\
\hline
\end{tabular}

\medskip

\textbf{Segreteria}

\medskip

\begin{tabular}{|c|c|}
\hline
Username & Password \\
\hline
segretario1 & segretario1\\
\hline
\end{tabular}

\medskip

\textbf{Amministratore}

\medskip


\begin{tabular}{|c|c|}
\hline
Username & Password \\
\hline
admin & admin\\
\hline
\end{tabular}

\medskip
\medskip

Senza considerare in campi adibiti all'autenticazione e l'identificazione, quali: matricola e password per \emph{studenti} e \emph{docenti}, username e password per \emph{Segreteria} e \emph{Amministratori}, il significato dei rimanenti campi è esplicato di seguito.

\medskip

Per quanto riguarda l'utenza \textbf{Studente}, i campi \emph{nome}, \emph{cognome} e \emph{data di nascita} rappresentano i dati anagrafici. 
Il campo \emph{ReputazioneTotale} rappresenta la reputazione totale acquisita dallo studente in seguito alle sue interazioni con le bacheche. In particolare, questo campo viene è determinato a partire dalla media di tutti i voti \emph{accordo} ricevuti.
I campi \emph{CFU} e \emph{Media} riguardano la sfera degli esami; il campo \emph{CFU} rappresenta il totale dei crediti formativi acquisiti dallo studente mentre il campo \emph{Media} è ricavato effettuando la media di tutte le votazioni conseguite.
Infine, il campo \emph{CorsoLaurea} indica il percorso di studi intrapreso dallo studente considerato.

\medskip

Per quanto riguarda l'utenza \textbf{Docente}, i campi \emph{nome}, \emph{cognome} rappresentano i dati anagrafici.

\subsection{Struttura file xml}

\subsubsection{Amministrazione}

\emph{FILE .XML}

\begin{lstlisting}[language=XML]
<?xml version="1.0" encoding="UTF-8"?>
<amministrazione xmlns:xsi="http://www.w3.org/2001/XMLSchema-instance" xsi:noNamespaceSchemaLocation="amministrazione.xsd">
    <amministratore>
        <username>admin</username>
        <password>admin</password>
    </amministratore>
</amministrazione>
\end{lstlisting}

\emph{FILE .XSD}

\begin{lstlisting}[language=XML]

<?xml version="1.0" encoding="UTF-8"?>
<xsd:schema xmlns:xsd="http://www.w3.org/2001/XMLSchema">

<xsd:element name="amministrazione">
 <xsd:complexType>
   <xsd:sequence>
    <xsd:element ref="amministratore" minOccurs="0" maxOccurs="unbounded" />
      </xsd:sequence>
   </xsd:complexType>
 </xsd:element>
<xsd:element name="amministratore">
 <xsd:complexType>
  <xsd:sequence>
   <xsd:element ref="username" minOccurs="1" maxOccurs="1"/>
   <xsd:element ref="password" minOccurs="1" maxOccurs="1"/>
  </xsd:sequence>
 </xsd:complexType>
</xsd:element>
<xsd:element name="username" type="xsd:string"/>
<xsd:element name="password" type="xsd:string"/>
</xsd:schema>
\end{lstlisting}

\subsubsection{Appartiene}

\begin{lstlisting}[language=XML]
<?xml version="1.0" encoding="UTF-8"?>
<appartenenze xmlns:xsi="http://www.w3.org/2001/XMLSchema-instance" xsi:noNamespaceSchemaLocation="appartiene.xsd">
  <appartenenza>
    <id>1</id>
    <idCorso>1</idCorso>
    <idCorsoDiLaurea>1</idCorsoDiLaurea>
  </appartenenza>
</appartenenze>
\end{lstlisting}

\subsubsection{Appelli}

\begin{lstlisting}[language=XML]
<?xml version="1.0" encoding="UTF-8"?>
<appelli xmlns:xsi="http://www.w3.org/2001/XMLSchema-instance" xsi:noNamespaceSchemaLocation="appelli.xsd">
    <appello>
        <id>1</id>
        <idCorso>1</idCorso>
        <dataOra>01/01/2000 10:10</dataOra>
    </appello>
</appelli>
\end{lstlisting}
\section{Definizioni XML}

\chapter{Manuale utente}

\chapter{Programmazione}

\chapter{Elenco utenze}

\end{document}
